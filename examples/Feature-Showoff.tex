\documentclass[
    ngerman,
    color=1b,
    % load_common, % läd oft benutzte Pakete, erhöht aber die Compile-Time drastisch
    submission,
    submission,
    dark_mode,
    % solution=true,
    boxarc,
    fleqn,
    % manual_term, % Restore original Functionality to term
    % shell_escape = false, % Kompatibilität mit sharelatex
    % T1,
]{rubos-tuda-template}

%%------------%%
%%--Packages--%%
%%------------%%
% (nicht in Class File um CompileTime zu verbessern)

\usepackage{forest}
\usepackage{tabularx}
\usepackage[color=accentcolor,textcolor=white,textwidth=.8cm,textsize=small,loadshadowlibrary]{todonotes}
\usepackage{tikz}

\usetikzlibrary{positioning}

%%-----------------------%%
%%--Abgabeeinstellungen--%%
%%-----------------------%%
% (Bis auf studentEins alle Optional)

\sheetnumber{1}
\groupnumber{69}
\addSubmittor{Ruben Deisenroth}{9876543}
\addSubmittor{Max Mustermann}{1234567}
\addSubmittor{Peter Peterson}{0000000}
\groupLeader{Senpai Yoda}
\semester{WiSe 2020/21}
% \version{2.1}
\fachbereich{Informatik}
\dozent{Prof. Dr. rer. nat. Foxy Schlaufux}
\date{\today}

%%----------------------------%%
%%--Stilistische Anpassungen--%%
%%----------------------------%%

% \renewcommand*{\taskformat}{H\thetask{}} % Taskprefix
% \renewcommand*{\subtaskformat}{\thesubtask\enskip} % This looks way cleaner
\termStyle{left-right-manual}
% \termOrder{printAuthor,printSheetNumber,printGroupNumber,printContributor,printVersion,,printSemester,printDate,printFachbereich}
\termLeft{printAuthor,printSubmittors}
\termRight{printSheetNumber,printVersion,printGroupNumber,printGroupLeader,printSemester}
\term{%
    \vspace{\baselineskip}%
    \printDate{}\hfill\printFachbereich{}%
}
\ConfigureHeadline{%
    headline={submittors-centered}
}

%%-------------------------%%
%%--Beginn des Dokumentes--%%
%%-------------------------%%

\begin{document}

    %%-----------%%
    %%--Titelei--%%
    %%-----------%%

    \title[Mathe 0]{Abgabe Hausübung\\im Fach\\ Mathematik 0}
    \maketitle{}
    %\printGradeTable{} % Erstellt eine Bewertungstabelle wie in Klausuren (Optionen für weglassen von Summe und Note Geplant)
    %\tableofcontents

    %%------------------------%%
    %%--Beginn der Hausübung--%%
    %%------------------------%%

    % Übung 1
    \begin{task}[points=8,solution=true]{\mdseries Systematisches Testen von Methoden}
        \begin{cpenumerate}[label=\alph*)]
            \item Was ist 1+1?
            im Fach \getShortTitle{} gehen wir mit Zahlen wie Folgt um:
            $1+1=-1-(-3)=\sqrt{4}=\underline{\underline{2}}$
            \item Was ist 2-1?

            $2-1=42-41=\sqrt{\left(\frac{2e^{42}}{\pi}\right)^{0}}=\underline{\underline{1}}$

        \end{cpenumerate}
    \end{task}
    % Übung 2
    \begin{task}[points=2,solution=true]{UwUOwO}
        LwL

        \begin{minipage}[t]{.5\textwidth-.5cm}
            \mbox{}
            %\begin{noindent}
        \begin{codeBlock}[autogobble,escapeinside=||,fontsize=\small]{minted language=java, title={\faCode{}\hfill ohne Generics\hfill\faCode}}
            public class StringContainer {
                public String value;
            }
            public class IntegerContainer {
                public Integer value;
            }
        \end{codeBlock}
        %\end{noindent}
        \end{minipage}%
        \hfill%
        \begin{minipage}[t]{.5\textwidth-.5cm}
            \mbox{}
            %\begin{noindent}
        \begin{codeBlock}[autogobble,escapeinside=||,fontsize=\small]{minted language=java, title={\faCode{}\hfill mit Generics\hfill\faCode}}
            public class GenericContainer<T> {
                public T value;
            }
        \end{codeBlock}
        %\end{noindent}
        \end{minipage}
    \end{task}
    \clearpage
    % Übung 3
    \begin{task}[points=auto,solution=true]{Alternativer style}
        Such pretty much wow
        \begin{subtask}[title={Ganzzahladdition auf \texorpdfstring{$\mathbb{N}$}{N}},points=3]
            Was ist 69+420?
            \begin{solution}
                Die Antwort auf Alles ist $42$. Die Antwort auf diese Frage ist jedoch \the\numexpr420+69\relax.
            \end{solution}
        \end{subtask}
        \begin{subtask}[title={Irgend son Graph},points=2]
            \begin{defBox}
                \gegeben $f(x)=0,5(x+1)^2-2$\\
                \zuberechnen Nullstellen von $f(x)$
            \end{defBox}
            \begin{solution}
                \begin{align*}
                    0,5(x+1)^2-2                                                    & =0                                                                                                 &  & \vert \text{Klammer auflösen}                        \\
                    0,5\cdot(x^2+2\cdot x \cdot 1 + 1^2)-2                          & =0                                                                                                 &  & \vert \text{Ausmultiplizieren}                       \\
                    0,5\cdot x^2+x +0,5-2                                           & =0                                                                                                                                                           \\
                    0,5\cdot x^2+x -1,5                                             & =0                                                                                                 &  & \vert \text{Mitternachtsformel mit a=0,5;b=1;c=-1,5} \\
                    \frac{-(1)\pm \sqrt{1^2-(4\cdot 0,5 \cdot (-1,5))}}{2\cdot 0,5} & =x_{1/2}                                                                                                                                                     \\
                    \frac{-1\pm \sqrt{1-(-3)}}{1}                                   & =x_{1/2}                                                                                                                                                     \\
                    \frac{-1\pm \sqrt{4}}{1}                                        & =x_{1/2}                                                                                                                                                     \\
                    -1 \pm 2                                                        & =x_{1/2} \Rightarrow \underline{\underline{x_1                                       = 1, x_2=-3}}
                \end{align*}
                \begin{grayInfoBox}
                    \antwort Die Funktion hat zwei Nullstellen, bei $x_1=1$ und bei $x_2=-1$.
                \end{grayInfoBox}
            \end{solution}
        \end{subtask}
    \end{task}
    \clearpage
    \begin{task}[points=2,solution=true]{Weitere Macros}
        Boxed/framed environments
        \begin{definition}[Mitternachtsformel] Für eine Polynom zweiten Grades in der Form $a\cdot x^2+b\cdot x + c$ \\gilt für $a,b,c \in \mathbb{R}$ immer:%

            \begin{align}
                x_{1/2} & =\frac{-b\pm\sqrt{b^{2}-4\cdot a \cdot c}}{2\cdot a}
            \end{align}

        \end{definition}
        \begin{definition}
            test
        \end{definition}
        % Math Fonts
        \verb+\mathscr+:$\mathscr{ABCDEFGHIJKLMNOPQRSTUVWXYZ}$

        \verb+\mathcal+:$\mathcal{ABCDEFGHIJKLMNOPQRSTUVWXYZ}$

        $\varphi, \psi$

        % Demonstration der gefixten Überschriftenhöhe
        \begin{minipage}[t]{.5\textwidth-1pt}
            \subsection{--------} % Nur Schmale Zeichen
            asdf
        \end{minipage}\hspace{2pt}%
        \begin{minipage}[t]{.5\textwidth-1pt}
            \subsection{\texorpdfstring{()MBbdI$\vert$}{()MBbdI|}} % Zeichen mit Maximaler Höhe
            asdf $\vcenter{\hbox{\tikz{\node[draw]{hi};}}}$
        \end{minipage}
    \end{task}
    \todo{TODO}
    \clearpage
    \begin{task}[points=0]{Punktetabellen (WIP)}

        \begin{itemize}
            \IfPointsLoadedF{\item ??}
                \mapTasks{
            \item \taskformat{}: \IfSubtasksTF{%
                    Gesamt \getPoints{\thetask} Punkte
                    \begin{itemize}
                        \mapSubtasks{\item \subtaskformat{}
                            = \getSubPoints{\the\value{task}}{\the\value{subtask}}}
                    \end{itemize}
                }{%
                    \getPoints{\thetask} Punkte
                }
                }
        \end{itemize}\par
        \begin{figure}[ht]
            \centering
            \begin{tikzpicture}
                \tikzstyle{gradetablenode}=[minimum size=1cm, draw]
                \tikzstyle{lightgraygradetablenode}=[gradetablenode, fill=fgcolor!10!\thepagecolor]
                \node[lightgraygradetablenode, text width=3cm, font=\sffamily, anchor=west] (n0-0) at (0, 1){Aufgabe};
                \node[gradetablenode, text width=3cm, font=\sffamily, below right = -\pgflinewidth and 0cm of n0-0.south west] (n1-0){Punkte (max)};
                \node[gradetablenode, text width=3cm, font=\sffamily, below right = -\pgflinewidth and 0cm of n1-0.south west] (n2-0){Punkte (erreicht)};
                \xdef\lasttask{0}
                \mapTasks{
                    \node[lightgraygradetablenode, font=\bfseries, right=-\pgflinewidth of n0-\the\numexpr\value{task}-1\relax.east] (n0-\the\value{task}) {\thetask{}};
                    \node[gradetablenode, right=-\pgflinewidth of n1-\the\numexpr\value{task}-1\relax.east] (n1-\the\value{task}) {\getPoints{\thetask}};
                    \node[gradetablenode, right=-\pgflinewidth of n2-\the\numexpr\value{task}-1\relax.east] (n2-\the\value{task}) {\IfSolutionT{\fatsf{\getPoints{\thetask}}}};
                    \xdef\lasttask{\numexpr\lasttask+1\relax}
                }
                % Probably + 1 in node names is too much, will check later
                \node[lightgraygradetablenode, minimum width=1.1cm, right=-\pgflinewidth of n0-\the\numexpr\lasttask\relax.east] (n0-\the\numexpr\lasttask + 1\relax){$\boldsymbol{\Sigma}$};
                \node[gradetablenode, minimum width=1.1cm, right=-\pgflinewidth of n1-\the\numexpr\lasttask\relax.east] (n1-\the\numexpr\lasttask + 1\relax){$\getPointsTotal{}$};
                \node[gradetablenode, minimum width=1.1cm, right=-\pgflinewidth of n2-\the\numexpr\lasttask\relax.east] (n2-\the\numexpr\lasttask + 1\relax){\IfSolutionT{\fatsf{\getPointsTotal{}}}};
                % % Extra lines
                \draw[transform canvas={xshift = -1mm}] (n0-0.north east) -- (n2-0.south east);
                \draw[transform canvas={xshift = 1mm}] (n0-\the\numexpr\lasttask+1.north west) -- (n2-\the\numexpr\lasttask+1.south west);
                \draw[thick] (n0-0.north west) rectangle (n2-\the\numexpr\lasttask+1.south east);
            \end{tikzpicture}
            \caption*{Punktetabelle, Design 2}
        \end{figure}
        \makeatletter
        \newcommand*{\currentname}{\@currentlabelname}
        \makeatother
        \begin{table}[ht]
            \centering
            \def\arraystretch{2}
            \rowcolors{2}{fgcolor!10!\thepagecolor}{fgcolor!10!\thepagecolor}
            \begin{tabularx}{\textwidth}{|X|c|c|}
                \toprule
                \fatsf{Aufgabe}                                                                      & \fatsf{möglich}             & \fatsf{erreicht}                                                      \\
                \midrule\midrule
                \mapTasks*{%
                    \textbf{\getTaskProperty{taskformat}:~\getTaskProperty{title}}                       & \getTaskProperty{points}    & \IfSolutionT{\fatsf{\getPoints{\thetask}}}\IfSubtasksTF{\mapSubtasks{ \\
                            \rowcolor{\thepagecolor}\getSubTaskProperty{subtaskformat}\getSubTaskProperty{title} & \getSubTaskProperty{points} & \IfSolutionT{\fatsf{\getSubTaskProperty{points}}}
                        }
                    }{}                                                                                                                                                                                        \\
                    \midrule\midrule
                }
                \textbf{Gesamt}                                                                      & \textbf{\getPointsTotal{}}  & \IfSolutionT{\fatsf{\getPointsTotal{}}}                               \\
                \bottomrule
            \end{tabularx}
            \caption*{Punktetabelle, Design 3}
            \label{tab:newtab}
        \end{table}
    \end{task}
\end{document}
