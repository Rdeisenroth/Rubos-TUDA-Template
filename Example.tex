\documentclass[ngerman]{tuda_hausuebung}
% Packages (nicht in Class File um CompileTime zu verbessern)
\usepackage[ngerman]{babel}
\usepackage{amsmath}
\usepackage{amssymb}
\usepackage{amsthm}
\usepackage{mathtools}
\usepackage{mleftright}
\usepackage{tikz}
\usetikzlibrary{patterns, shapes, intersections, arrows, math, decorations,decorations.pathreplacing, decorations.pathmorphing, positioning, calc, automata, chains, shapes.geometric, arrows.meta}
\usepackage{tcolorbox}
\tcbuselibrary{skins}
\usepackage{nccmath}
\RequirePackage[framemethod=tikz]{mdframed} % Required for creating the theorem, definition, exercise and corollary boxes
% Definitionen für das Dokument (Bis auf studentEins alle Optional)
\sheetnumber{1}
\groupnumber{69}
\studentEins{Ruben Deisenroth}{9876543}
\studentZwei{Max Mustermann}{1234567}
\studentDrei{Peter Peterson}{0000000}
\groupleader{Senpai Yoda}
\semester{WiSe 2020/21}
\fachbereich{Informatik}
\date{\textbf{\sffamily Datum:} \today}
\renewcommand*{\taskformat}{H\tasksep\thetask{}} % Taskprefix
\ConfigureHeadline{
	headline={submittors-centered}
}
\begin{document}
\title[Mathe 0]{Abgabe Hausübung\\im Fach\\ Mathematik 0}
\maketitle{}
%\printGradeTable{} % Erstellt eine Bewertungstabelle wie in Klausuren (Optionen für weglassen von Summe und Note Geplant)
%\tableofcontents

%% --Beginn Hausübung--%%
% Übung 1
\begin{task}[points=8,solution=true]{\mdseries Systematisches Testen von Methoden}
    \begin{cpenumerate}[label=\alph*)]
        \item Was ist 1+1?
        im Fach \getShortTitle{} gehen wir mit Zahlen wie Folgt um:
        $1+1=-1-(-3)=\sqrt{4}=\underline{\underline{2}}$
        \item Was ist 2-1?

        $2-1=42-41=\sqrt{\left(\frac{2e^{42}}{\pi}\right)^{0}}=\underline{\underline{1}}$

    \end{cpenumerate}
\end{task}
% Übung 2
\begin{task}[points=2,solution=true]{UwUOwO}
    LwL
\end{task}
\clearpage
% Übung 3
\begin{task}[points=5,solution=true]{Alternativer style}
    Such pretty much wow
    \begin{subtask}[title={Ganzzahladdition auf $\mathbb{N}$},points=3]
        Was ist 69+420?
        \begin{solution}
            Die Antwort auf Alles ist $42$. Die Antwort auf diese Frage ist jedoch \the\numexpr420+69\relax.
        \end{solution}
    \end{subtask}
    \begin{subtask}[title={Irgend son Graph},points=2]
        \gegeben $f(x)=0,5(x+1)^2-2$\\
        \zuberechnen Nullstellen von $f(x)$\\
        \begin{solution}
            \begin{align*}
                0,5(x+1)^2-2                                                    & =0                                                                                                 &  & \vert \text{Klammer auflösen}                        \\
                0,5\cdot(x^2+2\cdot x \cdot 1 + 1^2)-2                          & =0                                                                                                 &  & \vert \text{Ausmultiplizieren}                       \\
                0,5\cdot x^2+x +0,5-2                                           & =0                                                                                                                                                           \\
                0,5\cdot x^2+x -1,5                                             & =0                                                                                                 &  & \vert \text{Mitternachtsformel mit a=0,5;b=1;c=-1,5} \\
                \frac{-(1)\pm \sqrt{1^2-(4\cdot 0,5 \cdot (-1,5))}}{2\cdot 0,5} & =x_{1/2}                                                                                                                                                     \\
                \frac{-1\pm \sqrt{1-(-3)}}{1}                                   & =x_{1/2}                                                                                                                                                     \\
                \frac{-1\pm \sqrt{4}}{1}                                        & =x_{1/2}                                                                                                                                                     \\
                -1 \pm 2                                                        & =x_{1/2} \Rightarrow \underline{\underline{x_1                                       = 1, x_2=-3}}
            \end{align*}
        \end{solution}
    \end{subtask}
\end{task}
\clearpage
\begin{task}[points=2,solution=true]{Weitere Macros}
    asdfgg
    Boxed/framed environments
    \newtheoremstyle{ocrenumbox}% Theorem style name
    {0pt}% Space above
    {0pt}% Space below
    {\normalfont}% Body font
    {}% Indent amount
    {\small\bfseries\sffamily\color{accentcolor}}% Theorem head font
    {\;}% Punctuation after theorem head
    {0.25em}% Space after theorem head
    {
        \small\sffamily\color{accentcolor}\thmname{#1}\space\thmnumber{#2}% Theorem text (e.g. Theorem 2.1)
        \thmnote{\space\sffamily\bfseries\color{black}---\space#3}% Optional theorem note
    }

    \newmdenv[skipabove=7pt,
        skipbelow=7pt,
        rightline=false,
        leftline=true,
        topline=false,
        bottomline=false,
        backgroundcolor=accentcolor!10,
        linecolor=accentcolor,
        innerleftmargin=5pt,
        innerrightmargin=5pt,
        innertopmargin=5pt,
        innerbottommargin=5pt,
        leftmargin=0cm,
        rightmargin=0cm,
        linewidth=3pt,
        roundcorner=5pt]{dBox}%

    \newtcolorbox{def2}[1][]{
        %listing engine=minted, % Minted verwenden
        colback=accentcolor!10, %Hintergrundfarbe
        colframe=white, % Randfarbe
        arc=2pt,
        %listing only,  % Sonst will er den Plain Text nach dem Minted Listing noch anfügen
        %minted style=colorful, %Sieht actually worse aus imo
        %minted language=java, % Sprache setzen
        left=5pt, % Links Platz lassen
        enhanced, % Erlaubt uns, den ramen zu zeichnen
        fonttitle=\sffamily, % Titelschriftart auf 
        overlay={ % Für Grauen Bereich links
                \begin{tcbclipinterior}
                    \fill[accentcolor] (frame.south west) rectangle ([xshift=4pt]frame.north west); % Zeilennummernbereich färben
                \end{tcbclipinterior}
            },
        #1 % Weitere Argumente zulassen
    }
    \theoremstyle{ocrenumbox}%
    %\newenvironment{definition}{\begin{dBox}\begin{definitionT}}{\end{definitionT}\end{dBox}}%
    \newenvironment{definition}{\begin{def2}\begin{definitionT}}{\end{definitionT}\end{def2}}%
    \newtheorem{definitionT}{\textsf{Definition}}[task]%
    \begin{definition}[Mitternachtsformel] Für eine Polynom zweiten Grades in der Form $a\cdot x^2+b\cdot x + c$ \\gilt für $a,b,c \in \mathbb{R}$ immer:%
        \makeatletter
        \newcommand{\leqnomode}{\tagsleft@true}
        \newcommand{\reqnomode}{\tagsleft@false}
        \makeatother
        \leqnomode
        \begin{fleqn}[\parindent+2em]
            \begin{align}
                x_{1/2} & =\frac{-b\pm\sqrt{b^{2}-4\cdot a \cdot c}}{2\cdot a}
            \end{align}
        \end{fleqn}
    \end{definition}
    \begin{definition}
        test
    \end{definition}

    \begin{minipage}[t]{.5\textwidth}
        \subsection{--------} % Nur Schmale Zeichen
        asdf
    \end{minipage}
    \begin{minipage}[t]{.49\textwidth}
        \subsection{()MBbdI$\vert$} % Zeichen mit Maximaler Höhe
        asdf $\vcenter{\hbox{\tikz{\node[draw]{hi};}}}$
    \end{minipage}


    \begin{def2}
        a
    \end{def2}
\end{task}
\end{document}
