\documentclass[
    ngerman,
    color=1b,
    load_common,
    submission,
    % dark_mode,
    boxarc,
    % manual_term, % Restore original Functionality to term
    % shell_escape = false, % Kompatibilität mit sharelatex
]{rubos-tuda-template}
% Packages (nicht in Class File um CompileTime zu verbessern)
% Definitionen für das Dokument (Bis auf studentEins alle Optional)
\sheetnumber{1}
\groupnumber{69}
\addSubmittor{Ruben Deisenroth}{9876543}
\addSubmittor{Max Mustermann}{1234567}
\addSubmittor{Peter Peterson}{0000000}
\groupleader{Senpai Yoda}
\semester{WiSe 2020/21}
\fachbereich{Informatik}
\dozent{Prof. Dr. rer. nat. Foxy Schlaufux}
\date{\today}
\renewcommand*{\taskformat}{H\tasksep\thetask{}} % Taskprefix
\termStyle{left-right-manual}
% \termOrder{printAuthor,printSheetNumber,printGroupNumber,printContributor,printVersion,,printSemester,printDate,printFachbereich}
\termLeft{printAuthor,printSubmittors}
\termRight{printSheetNumber,printGroupNumber,printGroupLeader,printSemester}
\term{\printDate{}\hfill\printFachbereich{}}
\ConfigureHeadline{
	headline={submittors-centered}
}
\begin{document}
\title[Mathe 0]{Abgabe Hausübung\\im Fach\\ Mathematik 0}
\maketitle{}
%\printGradeTable{} % Erstellt eine Bewertungstabelle wie in Klausuren (Optionen für weglassen von Summe und Note Geplant)
%\tableofcontents

%% --Beginn Hausübung--%%

% Übung 1
\begin{task}[points=8,solution=true]{\mdseries Systematisches Testen von Methoden}
    \begin{cpenumerate}[label=\alph*)]
        \item Was ist 1+1?
        im Fach \getShortTitle{} gehen wir mit Zahlen wie Folgt um:
        $1+1=-1-(-3)=\sqrt{4}=\underline{\underline{2}}$
        \item Was ist 2-1?

        $2-1=42-41=\sqrt{\left(\frac{2e^{42}}{\pi}\right)^{0}}=\underline{\underline{1}}$

    \end{cpenumerate}
\end{task}
% Übung 2
\begin{task}[points=2,solution=true]{UwUOwO}
    LwL
\end{task}
\clearpage
% Übung 3
\begin{task}[points=auto,solution=true]{Alternativer style}
    Such pretty much wow
    \begin{subtask}[title={Ganzzahladdition auf \texorpdfstring{$\mathbb{N}$}{N}},points=3]
        Was ist 69+420?
        \begin{solution}
            Die Antwort auf Alles ist $42$. Die Antwort auf diese Frage ist jedoch \the\numexpr420+69\relax.
        \end{solution}
    \end{subtask}
    \begin{subtask}[title={Irgend son Graph},points=2]
        \begin{defBox}
            \gegeben $f(x)=0,5(x+1)^2-2$\\
            \zuberechnen Nullstellen von $f(x)$
        \end{defBox}
        \begin{solution}
            \begin{align*}
                0,5(x+1)^2-2                                                    & =0                                                                                                 &  & \vert \text{Klammer auflösen}                        \\
                0,5\cdot(x^2+2\cdot x \cdot 1 + 1^2)-2                          & =0                                                                                                 &  & \vert \text{Ausmultiplizieren}                       \\
                0,5\cdot x^2+x +0,5-2                                           & =0                                                                                                                                                           \\
                0,5\cdot x^2+x -1,5                                             & =0                                                                                                 &  & \vert \text{Mitternachtsformel mit a=0,5;b=1;c=-1,5} \\
                \frac{-(1)\pm \sqrt{1^2-(4\cdot 0,5 \cdot (-1,5))}}{2\cdot 0,5} & =x_{1/2}                                                                                                                                                     \\
                \frac{-1\pm \sqrt{1-(-3)}}{1}                                   & =x_{1/2}                                                                                                                                                     \\
                \frac{-1\pm \sqrt{4}}{1}                                        & =x_{1/2}                                                                                                                                                     \\
                -1 \pm 2                                                        & =x_{1/2} \Rightarrow \underline{\underline{x_1                                       = 1, x_2=-3}}
            \end{align*}
            \begin{grayInfoBox}
                \antwort Die Funktion hat zwei Nullstellen, bei $x_1=1$ und bei $x_2=-1$.
            \end{grayInfoBox}
        \end{solution}
    \end{subtask}
\end{task}
\clearpage
\begin{task}[points=2,solution=true]{Weitere Macros}
    asdfgg
    Boxed/framed environments
    \newtheoremstyle{ocrenumbox}% Theorem style name
    {0pt}% Space above
    {0pt}% Space below
    {\normalfont}% Body font
    {}% Indent amount
    {\small\bfseries\sffamily\color{accentcolor}}% Theorem head font
    {\;}% Punctuation after theorem head
    {0.25em}% Space after theorem head
    {
        \small\sffamily\color{accentcolor}\thmname{#1}\space\thmnumber{#2}% Theorem text (e.g. Theorem 2.1)
        \thmnote{\space\sffamily\bfseries\color{black}---\space#3}% Optional theorem note
    }
    \begin{definition}[Mitternachtsformel] Für eine Polynom zweiten Grades in der Form $a\cdot x^2+b\cdot x + c$ \\gilt für $a,b,c \in \mathbb{R}$ immer:%
        \makeatletter
        \newcommand{\leqnomode}{\tagsleft@true}
        \newcommand{\reqnomode}{\tagsleft@false}
        \makeatother
        \leqnomode
        \begin{fleqn}[\parindent+2em]
            \begin{align}
                x_{1/2} & =\frac{-b\pm\sqrt{b^{2}-4\cdot a \cdot c}}{2\cdot a}
            \end{align}
        \end{fleqn}
    \end{definition}
    \begin{definition}
        test
    \end{definition}

    % Demonstration der gefixten Überschriftenhöhe
    \begin{minipage}[t]{.5\textwidth}
        \subsection{--------} % Nur Schmale Zeichen
        asdf
    \end{minipage}
    \begin{minipage}[t]{.49\textwidth}
        \subsection{\texorpdfstring{()MBbdI$\vert$}{()MBbdI|}} % Zeichen mit Maximaler Höhe
        asdf $\vcenter{\hbox{\tikz{\node[draw]{hi};}}}$
    \end{minipage}
\end{task}\\

\begin{task}[points=0]{Punktetabellen (WIP)}

    \begin{itemize}
        \mapTasks{
        \item \taskformat{}: \IfSubtaskTF{%
                  Gesamt \getPoints{\thetask} Punkte
                  \begin{itemize}
                      \mapSubtasks{\item \subtaskformat{}
                            = \getSubPoints{\thetask}{\thesubtask}}
                  \end{itemize}
              }{%
                  \getPoints{\thetask} Punkte
              }
              }
    \end{itemize}\par

    \begin{tikzpicture}
        \tikzstyle{gradetablenode}=[minimum size=1cm, draw, thick]
        \tikzstyle{lightgraygradetablenode}=[gradetablenode, fill=fgcolor!10!\thepagecolor]
        \node[lightgraygradetablenode, text width=3cm, anchor=west] (n0-0) at (0, 1){Aufgabe};
        \node[gradetablenode, text width=3cm, below right = -\pgflinewidth and 0cm of n0-0.south west] (n1-0){Punkte (max)};
        \node[gradetablenode, text width=3cm, below right = -\pgflinewidth and 0cm of n1-0.south west] (n2-0){Punkte (erreicht)};
        \xdef\lasttask{1}
        \mapTasks{
            \node[lightgraygradetablenode, right=-\pgflinewidth of n0-\the\numexpr\value{task}-1\relax.east] (n0-\the\value{task}) {\thetask{}};
            \node[gradetablenode, right=-\pgflinewidth of n1-\the\numexpr\value{task}-1\relax.east] (n1-\the\value{task}) {\getPoints{\thetask}};
            \node[gradetablenode, right=-\pgflinewidth of n2-\the\numexpr\value{task}-1\relax.east] (n2-\the\value{task}) {};
            \xdef\lasttask{\value{task}}
        }
        % Probably + 1 in node names too much, will check later
        \node[lightgraygradetablenode, minimum width=1.1cm, right=-\pgflinewidth of n0-\the\numexpr\lasttask-1\relax.east] (n0-\the\numexpr\lasttask + 1\relax){$\Sigma$};
        \node[gradetablenode, minimum width=1.1cm, right=-\pgflinewidth of n1-\the\numexpr\lasttask-1\relax.east] (n1-\the\numexpr\lasttask + 1\relax){\getPointsTotal{}};
        \node[gradetablenode, minimum width=1.1cm, right=-\pgflinewidth of n2-\the\numexpr\lasttask-1\relax.east] (n2-\the\numexpr\lasttask + 1\relax){};
        % Extra lines
        \draw[thick,transform canvas={xshift = -1mm}] (n0-0.north east) -- (n2-0.south east);
        \draw[thick,transform canvas={xshift = 1mm}] (n0-\the\numexpr\lasttask+1.north west) -- (n2-\the\numexpr\lasttask+1.south west);
    \end{tikzpicture}
\end{task}
\end{document}
