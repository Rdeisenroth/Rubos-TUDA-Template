\documentclass[ngerman]{Tuda_Hausuebung}
% Packages (nicht in Class File um CompileTime zu verbessern)
\usepackage[utf8]{inputenc}
\usepackage[T1]{fontenc}
\usepackage[ngerman]{babel}
\usepackage{amsmath}
\usepackage{amssymb}
\usepackage{amsthm}
\usepackage{mathtools}
\usepackage{mleftright}
\usepackage{tikz}
\usetikzlibrary{patterns, shapes, intersections, arrows, math, decorations,decorations.pathreplacing, decorations.pathmorphing, positioning, calc, automata, chains, shapes.geometric, arrows.meta}
\usepackage{tcolorbox}

% Definitionen für das Dokument
\sheetnumber{1}
\groupnumber{69}
\studentEins{Ruben Deisenroth}{9876543}
\studentZwei{Max Mustermann}{1234567}
\studentDrei{Peter Peterson}{0000000}
\groupleader{Senpai Yoda}
\date{\textbf{\sffamily Datum:} \today}
\renewcommand*{\taskformat}{H\tasksep\thetask{}} % Taskprefix
\begin{document}
\title[Mathe 0]{Abgabe Hausübung\\im Fach\\ Mathematik 0}
\maketitle{}
\printGradeTable{} % Erstellt eine Bewertungstabelle wie in Klausuren (Optionen für weglassen von Summe und Note Geplant)
%\tableofcontents

%% --Beginn Hausübung--%%
% Übung 1
\begin{task}[points=8,solution=true]{\mdseries Systematisches Testen von Methoden}
    \begin{cpenumerate}[label=\alph*)]
        \item Was ist 1+1?
        im Fach \getShortTitle{} gehen wir mit Zahlen wie Folgt um:
        $1+1=-1-(-3)=\sqrt{4}=\underline{\underline{2}}$
        \item Was ist 2-1?

        $2-1=42-41=\sqrt{\left(\frac{2e^{42}}{\pi}\right)^{0}}=\underline{\underline{1}}$

    \end{cpenumerate}
\end{task}
% Übung 2
\begin{task}[points=2,solution=true]{UwUOwO}
    b
\end{task}
\clearpage
% Übung 3
\begin{task}[points=5,solution=true]{Alternativer style}
    Such pretty much wow
    \begin{subtask}[title={Ganzzahladdition auf $\mathbb{N}$},points=2]
        \gegeben $f(x)=0,5(x+1)^2-2$\\
        \zuberechnen Nullstellen von $f(x)$\\
        \begin{solution}
            \begin{align*}
                0,5(x+1)^2-2                                                    & =0                                                                                                 &  & \vert \text{Klammer auflösen}                        \\
                0,5\cdot(x^2+2\cdot x \cdot 1 + 1^2)-2                          & =0                                                                                                 &  & \vert \text{Ausmultiplizieren}                       \\
                0,5\cdot x^2+x +0,5-2                                           & =0                                                                                                                                                           \\
                0,5\cdot x^2+x -1,5                                             & =0                                                                                                 &  & \vert \text{Mitternachtsformel mit a=0,5;b=1;c=-1,5} \\
                \frac{-(1)\pm \sqrt{1^2-(4\cdot 0,5 \cdot (-1,5))}}{2\cdot 0,5} & =x_{1/2}                                                                                                                                                     \\
                \frac{-1\pm \sqrt{1-(-3)}}{1}                                   & =x_{1/2}                                                                                                                                                     \\
                \frac{-1\pm \sqrt{4}}{1}                                        & =x_{1/2}                                                                                                                                                     \\
                -1 \pm 2                                                        & =x_{1/2} \Rightarrow \underline{\underline{x_1                                       = 1, x_2=-3}}
            \end{align*}
        \end{solution}
    \end{subtask}
\end{task}

\end{document}
